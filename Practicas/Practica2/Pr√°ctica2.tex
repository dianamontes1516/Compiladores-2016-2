% Created 2016-09-08 jue 16:13
\documentclass[11pt]{article}
\usepackage[utf8]{inputenc}
\usepackage[T1]{fontenc}
\usepackage{fixltx2e}
\usepackage{graphicx}
\usepackage{grffile}
\usepackage{longtable}
\usepackage{wrapfig}
\usepackage{rotating}
\usepackage[normalem]{ulem}
\usepackage{amsmath}
\usepackage{textcomp}
\usepackage{amssymb}
\usepackage{capt-of}
\usepackage{hyperref}
\author{Elisa Viso Gurovich}
\date{\today}
\title{Práctica 2: Analizadores sintácticos.\\\medskip
\large Compiladores 2017-1}
\hypersetup{
 pdfauthor={Elisa Viso Gurovich},
 pdftitle={Práctica 2: Analizadores sintácticos.},
 pdfkeywords={},
 pdfsubject={},
 pdfcreator={Emacs 24.5.1 (Org mode 8.3.4)}, 
 pdflang={English}}
\begin{document}

\maketitle

\section{Analizadores sintácticos descendentes (\emph{Top-Down})}
\label{sec:orgheadline1}
\begin{itemize}
\item Ejercicios teóricos:
\begin{enumerate}
\item ¿Cuales són las maneras de implementar un analizador sintáctico descendente?
\item ¿Que carácterísticas debe de cumplir una gramática libre del contexto para
que pueden tener un reconocedor descendente recursivo sin caer en ciclos 
ni hacer backtrack?
\item Implementa un anlizador sintáctico para la siguiente gramática:
\begin{verbatim}
S -> Aa | b
A -> Ac | Sd | epsilon
\end{verbatim}
No debe caer en ciclos ni hacer backtrack.
\end{enumerate}
\end{itemize}


\section{Analizadores Sintácticos ascendentes (\emph{Bottom-UP})}
\label{sec:orgheadline2}
\begin{itemize}
\item Instalación de \textbf{Byacc/J}
\begin{enumerate}
\item Descargar binario de \url{http://byaccj.sourceforge.net/#download}
\item Instalar bibliotecas de compatibilidad con software para arquitectura de 32 bits.
\url{http://askubuntu.com/questions/454253/how-to-run-32-bit-app-in-ubuntu-64-bit} . 
En algunos sistemas operativos ya están instaladas.
\item Descomprimir tar.gz
\item Hasta aquí ya tiene un binario que puede ser ejecutado como cualquier otro binario.
La sugerencia es pasarlo a algún lugar en el PATH para que puedan ejecutarlo desde
cualquier ruta. 
\begin{verbatim}
$ cp yacc.linux /usr/bin/byacc
$ byacc
\end{verbatim}
\end{enumerate}

\item Ejercicios:
\begin{enumerate}
\item Implementa un intérprete para cada una de las siguientes dos gramáticas. Utiliza
\textbf{byacc} y \textbf{jflex}.
\begin{itemize}
\item Gramática 1 
\begin{verbatim}
E -> E + T
    | E - T
    | T
T -> T * F
    | T / F
    | F
 F -> NUMBER   
     | - NUMBER
\end{verbatim}
\item Gramatica 2
\begin{verbatim}
E -> T + E
    | T - E
    | T
T -> F * T
    | F / T
    | F
 F -> NUMBER   
     | - NUMBER
\end{verbatim}
\end{itemize}
\item Encuentra a manera de imprimir la pilas de reconocimiento cada que se hace una reducción.
\item ¿Qué resultado da la evaluación de la expresión \textbf{3-2+8}? Explica el motivo de los resultados.
\end{enumerate}
\end{itemize}

\section{Administrativos:}
\label{sec:orgheadline3}
\begin{enumerate}
\item Se deberá crear una nueva carpeta \textbf{REPO/Prácticas/Práctica2} con los siguientes
archivos.
\begin{itemize}
\item tokens.l : script que hará el analizador léxico.
\item arith-left.y : script con las reglas de la gramática 1.
\item arith-rigth.y : script con las reglas de la gramática 2.
\item make: tendrá dos tags right y left. Permitirá introducir un archivo para prueba.
\item README.md : archivo que contendrá las respuestas de los ejercicios teóricos.
\end{itemize}
\item Se entregará antes del 23.09.16
\end{enumerate}
\end{document}
